\section{Automi a Stati Finiti}

\subsection{Automa a Stati Finiti Deterministico (DFA)}
Un automa a stati finiti deterministico (DFA) è definito formalmente come una quintupla $A = (Q, \Sigma, \delta, q_0, F)$:
\begin{itemize}
    \item $Q$: un insieme finito di stati.
    \item $\Sigma$: un insieme finito di simboli di input (l'alfabeto).
    \item $\delta$: la funzione di transizione, definita come $\delta: (Q \times \Sigma) \rightarrow Q$. Per una coppia (stato, simbolo) restituisce un singolo stato di arrivo (es. $\delta(q_i, a) = q_j$).
    \item $q_0 \in Q$: lo stato iniziale.
    \item $F \subseteq Q$: l'insieme degli stati finali o di accettazione.
\end{itemize}

\subsection{Linguaggio Accettato e Configurazione}
Il linguaggio accettato da un automa A, denotato con $L(A)$, è l'insieme di tutte le stringhe $w$ che vengono accettate dall'automa.
\[ L(A) = \{ w \in \Sigma^* \mid w \text{ è accettata da A} \} \]
Una \textbf{configurazione istantanea} descrive lo stato dell'automa in un dato momento e consiste in una coppia `[stato attuale, stringa ancora da leggere]`.
\begin{itemize}
    \item Esempio: $[q_i, aw] \rightarrow [q_j, w]$ se $\delta(q_i, a) = q_j$.
\end{itemize}

\subsection{Esempio Pratico: Riconoscimento di "11"}
Consideriamo un automa che riconosce stringhe sull'alfabeto $\{0, 1\}$ che contengono due '1' consecutivi.
\begin{itemize}
    \item $Q = \{q_0, q_1, q_2\}$
    \item $\Sigma = \{0, 1\}$
    \item $q_0$ è lo stato iniziale.
    \item $F = \{q_2\}$
    \item La funzione di transizione $\delta$ è definita dalla seguente tabella:
\end{itemize}

\begin{center}
\begin{tabular}{|c|c|c|}
    \hline
    \textbf{Stato} & \textbf{0} & \textbf{1} \\
    \hline
    $q_0$ & $q_0$ & $q_1$ \\
    \hline
    $q_1$ & $q_0$ & $q_2$ \\
    \hline
    $q_2$ & $q_2$ & $q_2$ \\
    \hline
\end{tabular}
\end{center}

\textbf{Computazione per la stringa $w = 10110$:}
\[ [q_0, 10110] \rightarrow [q_1, 0110] \rightarrow [q_0, 110] \rightarrow [q_1, 10] \rightarrow [q_2, 0] \rightarrow [q_2, \epsilon] \]
Poiché l'automa termina nello stato $q_2$, che è uno stato finale, la stringa $w=10110$ è accettata.

\begin{center}
    \begin{tikzpicture}[shorten >=1pt, node distance=2.5cm, on grid, auto] 
       \node[state, initial] (S) {S}; 
       \node[state] (A) [right=of S] {A}; 
       \node[state] (B) [right=of A] {B};
       \node[state, accepting] (C) [right=of B] {C};
       
       \path[->] 
        (S) edge [loop above] node {b} (S)
            edge node {a} (A)
        (A) edge node {a} (B)
            edge [bend left=60] node {b} (S)
        (B) edge node {a} (C)
            edge [bend left=60] node {b} (S)
        (C) edge [loop above] node {a,b} (C);
    \end{tikzpicture}
    \end{center}

\subsection{Complemento di un Linguaggio}
\textbf{Teorema:} Se $A = (Q, \Sigma, \delta, q_0, F)$ è un DFA che accetta il linguaggio $L(A)$, allora l'automa $A' = (Q, \Sigma, \delta, q_0, Q-F)$ accetta il linguaggio complemento $\Sigma^* - L(A)$.
\begin{itemize}
    \item In pratica, per ottenere un automa che accetta il linguaggio complemento, è sufficiente scambiare gli stati finali con quelli non finali.
\end{itemize}

\subsection{Esempio: Riconoscimento di Stringhe con Parità Pari}
Consideriamo un automa per riconoscere il linguaggio delle stringhe su $\{a,b\}$ con un numero pari di 'a' e un numero pari di 'b'.

\subsubsection{Definizione dei Non-Terminali/Stati}
I non-terminali della grammatica corrispondono agli stati dell'automa e tengono traccia della parità dei simboli letti.
\begin{itemize}
    \item \textbf{S}: numero pari di 'a' e numero pari di 'b' (stato iniziale e finale).
    \item \textbf{A}: numero dispari di 'a' e numero pari di 'b'.
    \item \textbf{B}: numero pari di 'a' e numero dispari di 'b'.
    \item \textbf{C}: numero dispari di 'a' e numero dispari di 'b'.
\end{itemize}

\subsubsection{Diagramma di Transizione}
Il diagramma mostra come l'automa cambia stato in base al simbolo letto.

\begin{center}
\begin{tikzpicture}[shorten >=1pt, node distance=3cm, on grid, auto] 
   \node[state, initial, accepting] (S) {S}; 
   \node[state] (A) [above right=of S] {A}; 
   \node[state] (B) [below right=of S] {B};
   \node[state] (C) [below right=of A] {C};
   
   \path[->] 
    (S) edge [bend left] node {a} (A)
        edge [bend right] node [swap] {b} (B)
    (A) edge [bend left] node {a} (S)
        edge node {b} (C)
    (B) edge [bend right] node [swap] {a} (C)
        edge [bend right] node [swap] {b} (S)
    (C) edge node {b} (A)
        edge [bend right] node [swap] {a} (B);
\end{tikzpicture}
\end{center}

\subsubsection{Funzione di Transizione per NFA}
\[ \delta: Q \times (\Sigma \cup \{\epsilon\}) \rightarrow 2^Q \]
Questo significa che per uno stato in $Q$ e un simbolo (o $\epsilon$), la funzione restituisce un \textit{insieme} di possibili stati successivi (indicato da $2^Q$, l'insieme delle parti di Q). 
